\documentclass[12pt]{article}
\usepackage{enumitem}
\usepackage{array}
\renewcommand{\thesubsection}{\alph{subsection}.}

\begin{document}
\title{Assignment \#5}
\author{Yoav Zimmerman (304125151) \\
	    CS 161: Artificial Intelligence}
\maketitle


\begin{enumerate}

	% #1
	\item We use truth tables to show pairs of sentences are equivalent. The bold columns in the following tables represent sentences that are logically equivalent. \\ 
	\\ 
	\begin{tabular}{ | c | c | >{\bf}c | >{\bf}c | }
	\hline
	P & Q & P $\Rightarrow \neg$ Q & Q $\Rightarrow \neg$ P \\
	\hline 
	T & T & F & F \\
	\hline 
	T & F & T & T \\
	\hline 
	F & T & T & T \\
	\hline 
	F & F & T & T \\
	\hline
	\end{tabular}
	
	\begin{tabular}{ | c | c | >{\bf}c | c | c | >{\bf}c |}
	\hline
	P & Q & P $\Leftrightarrow \neg$ Q & P $\land \neg$ Q & $\neg$ P $\land$ Q & ((P $\land \neg$ Q) $\lor$ ( $\neg$ P $\land$ Q)) \\
	\hline 
	T & T & F & F & F & F \\
	\hline 
	T & F & T & T & F & T \\
	\hline 
	F & T & T & F & T & T \\
	\hline 
	F & F & F & F & F & F \\
	\hline
	\end{tabular}
	\\
	
	% #2
	\item 
	\begin{enumerate}
		\item (Smoke $\Rightarrow$ Fire) $\Rightarrow$ ($\neg$ Smoke $\Rightarrow \neg$ Fire) 
		\\
		\\
		\begin{tabular}{ | c | c | c | c | >{\bf}c |}
		\hline
		Smoke & Fire & (Smoke $\Rightarrow$ Fire) & ($\neg$ Smoke $\Rightarrow \neg$ Fire) & Final \\
		\hline 
		T & T & T & T & T \\
		\hline 
		T & F & F & T & T \\
		\hline 
		F & T & T & F & F \\
		\hline 
		F & F & T & T & T \\
		\hline
		\end{tabular}
		\\
		
		Since the meaning of the sentence is neither empty nor complete, it is \textbf{neither} valid nor unsatisfiable.
		
		\item (Smoke $\Rightarrow$ Fire) $\Rightarrow$ ((Smoke $\lor$ Heat) $\Rightarrow$ Fire) 
		\\
		\\
		\begin{tabular}{ | c | c | c | c | c | >{\bf}c |}
		\hline
		Smoke & Fire & Heat & (Smoke $\Rightarrow$ Fire) & ((Smoke $\lor$ Heat) $\Rightarrow$ Fire) & Final  \\
		\hline 
		T & T & T & T & T & T \\
		\hline 
		T & T & F & T & T & T \\
		\hline 
		T & F & T & F & F & T \\
		\hline 
		T & F & F & F & F & T \\
		\hline
		F & T & T & T & T & T \\
		\hline 
		F & T & F & T & T & T \\
		\hline 
		F & F & T & T & F & F \\
		\hline 
		F & F & F & T & T & T \\
		\hline
		\end{tabular}
		\\
		
		Since the meaning of the sentence is neither empty nor complete, it is \textbf{neither} valid nor unsatisfiable.
		
		\item ((Smoke $\land$ Heat) $\Rightarrow$ Fire) $\iff$  \\
			((Smoke $\Rightarrow$ Fire) $\lor$ (Heat $\Rightarrow$ Fire)) 
		
		\begin{tabular}{ | c | c | c | c |}
		\hline
		Smoke & Fire & Heat & ((Smoke $\land$ Heat) $\Rightarrow$ Fire)  \\
		\hline 
		T & T & T & T  \\
		\hline 
		T & T & F & T  \\
		\hline 
		T & F & T & F  \\
		\hline 
		T & F & F & T  \\
		\hline 
		F & T & T & T  \\
		\hline 
		F & T & F & T  \\
		\hline 
		F & F & T & T  \\
		\hline 
		F & F & F & T  \\
		\hline 
		\end{tabular}
		\\

		\begin{tabular}{ | c | c | c | >{\bf}c |}
		\hline
		(Smoke $\Rightarrow$ Fire) & (Heat $\Rightarrow$ Fire) & ((Smoke $\Rightarrow$ Fire) $\lor$ (Heat $\Rightarrow$ Fire)) & Final  \\
		\hline 
		T & T & T & T  \\
		\hline 
		T & T & T & T  \\
		\hline 
		F & F & F & T \\
		\hline 
		F & T & T & T  \\
		\hline 
		T & T & T & T  \\
		\hline 
		T & T & T & T  \\
		\hline 
		T & F & T & T  \\
		\hline 
		T & T & T & T  \\
		\hline 
		\end{tabular}
		\\
		
		Since the meaning of the sentence is a complete set, it is \textbf{valid}.
	\end{enumerate}
	
	% # 3
	\item
	\begin{enumerate}
		\item To encode the information in propositional logic, we first have to decide on boolean variables: \textit{Mythical, Mortal, Mammal, Horned, Magical}. We then write the rules using logical notation:
		\begin{enumerate}
			\item Mythical $\Rightarrow \neg$ Mortal
			\item $\neg$ Mythical $\Rightarrow$ (Mortal $\land$ Mammal)
			\item ($\neg$ Mortal $\lor$ Mammal) $\Rightarrow$ Horned
			\item Horned $\Rightarrow$ Magical
		\end{enumerate}
		
		\item The above statements can also be written in Conjunctive Normal Form (CNF): \\ \\
		($\neg$ Mythical $\lor$ $\neg$ Mortal) \\
		$\land$ \\
		(Mythical $\lor$ Mortal) \\
		$\land$ \\
		(Mythical $\lor$ Mammal) \\
		$\land$ \\
		(Mortal $\lor$ Horned) \\
		$\land$ \\
		(Mortal $\lor$ $\neg$ Mammal) \\
		$\land$ \\
		($\neg$ Horned $\lor$ Magical) \\
		
		\item To prove that the unicorn is \textit{Horned}, we follow the derivation:
		\begin{enumerate}
			\item Mythical $\lor$ $\neg$ Mythical \textit{(always true)}
			\item $\neg$ Mortal $\lor$ (Mortal $\land$ Mammal) \textit{(sentences i. and ii.)}
			\item ($\neg$ Mortal $\lor$ Mortal) $\land$ ($\neg$ Mortal $\lor$ Mammal) \textit{(distribution)}
			\item $\neg$ Mortal $\lor$ Mammal \textit{(left term is always true)}
			\item \textbf{Horned} \textit{(sentence iii.)}
		\end{enumerate}
		To prove that the unicorn is \textit{Magical} is trivial once we have proved that the unicorn is \textit{Horned} by applying \textit{sentence iv}. It is not possible to prove that the unicorn is \textit{Mythical}.
	\end{enumerate}
	
	
		
\end{enumerate}


\end{document}